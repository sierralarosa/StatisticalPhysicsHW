\documentclass[a4paper,floatfix,nofootinbib]{article}
\usepackage{amsmath,amssymb}
\usepackage{graphicx}
\usepackage{listings}
\graphicspath{ { . } }

\title{Homework 2}
\author{Sierra LaRosa}

\begin{document}

\maketitle

\section*{Problem 2.24}
For a \textit{large} two-state paramagnet the multiplicity function is very sharply peaked about $N_{\uparrow} = N/2$.

\subsection*{a}
Use stirling's approximation to estimate the height of the peak in the multiplicity function.

First, define the multiplicity function for a two-state paramagnet:
\begin{equation*}
    \Omega(N_{\uparrow}) = \binom{N}{N_{\uparrow}}
\end{equation*}

Which as defined by the textbook in equation (2.7), the following can be concluded:
\begin{equation*}
    \Omega_{max} = \frac{N!}{N_{\uparrow} ! N_{\downarrow} ! } 
\end{equation*}
And from the definition of $N_{\uparrow}$ it is shown that: $N_{\downarrow} = N - N/2 = N/2$. When using Stirling's approximation ($N! = N^N e^{-N} \sqrt{2 \pi N} $) is used for the previous equation, it is obvious to see that the following is obtained:
\begin{align*}
    \Omega_{max} &= \frac{N^N e^{-N} \sqrt{2 \pi N}}{{{N/2}^{N/2} e^{-{N/2}} \sqrt{2 \pi {N/2}}}^2} \\
    &= \frac{N^N e^{-N} \sqrt{2 \pi N}}{{N/2}^N e^{-N} \pi N} \\
    &= \frac{N^N e^{-N} \sqrt{2 \pi N}}{\pi N \frac{N^N}{2^N} e^{-N} } \\
    &= 2^N \frac{ \sqrt{2 \pi N}}{\pi N } \\
    \therefore \Omega_{max} &= 2^N \frac{ 2 }{\sqrt{ \pi N }}
\end{align*}

\subsection*{b}
Use the methods of section 2.4 in the textbook to derive a formula for the multiplicity function fo the multiplicity function in the vicinity fo the peak in terms of $x = N_{\uparrow} - \frac{N}{2}$. And check that the formula agrees with \textbf{Problem 2.24a}'s solution. i.e. {x = 0} 

First, it would be beneficial to describe a parameter about the $N_{\uparrow}$ or $\eta = x + \frac{N}{2}$.

Now using the same idea as described in \textbf{Problem 2.24a}'s solution, the following can be shown.
\begin{align*}
   \Omega &= \frac{N!}{\eta ! \left( N - \eta \right) \ ! } \\
   &= \frac{N!}{\left(x + \frac{N}{2} \right) ! \left( N - \left(x + \frac{N}{2} \right) \right) \ ! } \\
    &= \frac{N!}{\left(x + \frac{N}{2} \right) ! \left( \frac{N}{2} - x \right) \ ! }
\end{align*}
And applying Stirling's approximation:
\begin{align*}
    \Omega &= \frac{N^N e^{-N} \sqrt{2 \pi N}}{ \left(x + \frac{N}{2}\right)^\left(x + \frac{N}{2}\right) e^{-\left(x + \frac{N}{2}\right)} \sqrt{2 \pi \left(x + \frac{N}{2}\right)} \times \left( \frac{N}{2} - x \right)^\left( \frac{N}{2} - x \right) e^{-\left( \frac{N}{2} - x \right)} \sqrt{2 \pi \left( \frac{N}{2} - x \right)} } \\
    &= \frac{N^N}{{\left( x + \frac{N}{2} \right)}^{x + \frac{N}{2}} \multiply {\left( \frac{N}{2} - x \right)}^{ \frac{N}{2} - x } } \multiply \sqrt{\frac{N}{2 \pi \left(x + \frac{N}{2} \right) \left( \frac{N}{2} - x \right) }} \\
    &= \frac{N^N}{{\left( x + \frac{N}{2} \right)}^{x + \frac{N}{2}} \multiply {\left( \frac{N}{2} - x \right)}^{ \frac{N}{2} - x } } \multiply \sqrt{\frac{N}{2 \pi \left( \frac{N^2}{4} - x^2 \right) }} \\
    &= \frac{N^N}{{\left( x + \frac{N}{2} \right)}^{\frac{N}{2}} \multiply {\left( x + \frac{N}{2} \right)}^{x} \multiply {\left( \frac{N}{2} - x \right)}^{ \frac{N}{2} } \multiply {\left( \frac{N}{2} - x \right)}^{ -x } } \multiply \sqrt{\frac{N}{2 \pi \left( \frac{N^2}{4} - x^2 \right) }} \\
    &= \frac{N^N}{{\left( x + \frac{N}{2} \right)}^{\frac{N}{2}} \multiply {\left( x + \frac{N}{2} \right)}^{x} \multiply {\left( \frac{N}{2} - x \right)}^{ \frac{N}{2} } \multiply {\left( \frac{N}{2} - x \right)}^{ -x } } \multiply \sqrt{\frac{N}{2 \pi \left( \frac{N^2}{4} - x^2 \right) }} \\
    \ln{\Omega} &= N\ln(N) - \frac{N}{2} \ln \left[ \frac{N^2}{4} - x^2 \right] - x \ln \left[ x + \frac{N}{2} \right] + x \ln \left[ \frac{N}{2} - x \right] + \ln \sqrt{\frac{N}{2 \pi}} - \frac{1}{2} \ln \left(\frac{N^2}{4} - x^2 \right)
\end{align*}
Now it would be beneficial to the solution that the subsititutions:
\begin{align*}
    ln \left[ {\frac{N}{2}}^2 - x^2 \right] &= \ln \left( { \frac{N}{2} }^2 \right) + \ln  \left[ 1 - {\left( \frac{2x}{N} \right)}^2 \right]
    &= 2 \ln \left( \frac{N}{2} \right) + {\left( \frac{2x}{N} \right)}^2
\end{align*}
, and
\begin{align*} 
    ln \left[ \frac{N}{2} \pm \right] &= \ln \left( \frac{N}{2} \right) + \ln \left( 1 \pm \frac{2x}{N} \right)
    &= \ln \left( \frac{N}{2} \right) \pm \frac{2x}{N}
\end{align*}
(The precident for these jumps is dictated in the work done in the set of equations 2.19 in the textbook).
Now pluggin in the previous subsititutions in to the previous work:
\begin{align*}
    \ln \Omega &= N \ln N \ - \ N \ln \frac{N}{2} \ + \ \frac{2 x^2}{N} - x \ln \frac{N}{2} - \frac{2 x^2}{N} + x \ln \frac{N}{2} - \frac{2 x^2}{N} + \ln \sqrt{\frac{N}{2 \pi}} - \ln \frac{N}{2} + \frac{2 x^2}{N^2}
    &= N \ln 2 \ + \ \frac{2 x^2}{N} + \ln \sqrt{\frac{N}{2 \pi}} + \frac{2 x^2}{N^2}
    \therefore \Omega &= 2^N \sqrt{\frac{2}{\pi N}} e^{2x^2 / N}
\end{align*}
This results in a Gaussian function as described in the textbook. And as $x \rightarrow 0$ the closer the answer to \textbf{Problem 2.24a} matches with this result!

\subsection*{c}
How wide is the multiplicity function?

Simple: any Gaussian function falls off of its peak at ${\Omega}_{\pi} = \frac{{\Omega}_max}{e}$. Here that value would be defined as the $\frac{2x_{\pi}^2}{N} = 1$ which becomes $x_{\pi} = \sqrt{\frac{N}{2}}$ or the full width of the peak is $\sqrt{2N}$

\subsection*{d}
How likely are the macrostates of 1 million coin tosses?

Before diving into the specifics: assume that the macrostate $n$H and $N-n$T coin toses can be described simply as: $m = N_H - N_T = N$ 

From recitation a hint was given that:
\begin{equation*}
    P_N(m) = {\left( 2 \pi N p q \right)}^{-1/2} e^{- \frac{m - {\leftangle m \rightangle}^2}{8 N p q }}
\end{equation*}

Which can be reduced trivially to:
\begin{equation*}
    = \sqrt{\frac{2}{\pi N}} e^{\frac{-m^2}{2 N}}
\end{equation*}

Now, the difference between the macro states described are $2,000$ for a macrostate with $501,000$ Heads $C_1$ and $20,000$ for a macrostate with $510,000$ Heads $C_2$. As a result: It is clear that $C_1$ is much more likely to happen, as the height of the magnitude of $C_1$ is below $1/e$ where as $C_2$ is calculated to be above the peak calculated using the reduced equation.

$\therefore$ the case $C_1$ is significantly more likely to happen than that of $C_2$.

\section{Problem 2.23}
Consider a two state paramagnet with $10^23$ elementary dipoles, with total energy fixed at zero so that exactly half of the dipoles are parallelly aligned and the other half are anitparallelly aligned.

\subsection{a}
How many microstates accessible witht the macrostate describe.
\begin{align*}
    \Omega &= \binom{N}{N/2}
    &= \frac{N^N e^{-N} \sqrt{2 \pi N}}{{\left( {N/2}^{N/2} e^{-N/2} \sqrt{\pi N} \right)}^2}
    &= 2^N \sqrt{ \frac{2}{\pi N} } 
\end{align*}
Because $N$ is of the order $10^23$ and because there are $2^{10^23}$ and this number is significantly larger than any number that is on the scale of Avagadro's number. Therefore the number of microstates of dipoles being aligned in parallel are about $2^{10^23}$.

\subsection{b}
Suppose that the microstate of this system changes a billion times per second. How many microstates will the system explore in ten billion years?

The universe is $3 \multiply 10^{17}$, therefore to calculate how many states that this system can traverse if the system suffles randomly every nanosecond is $3 \multiply 10^{24}$ i.e. the number of states that the system will explore

\subsection{c}
Is it correct to say that, if you wait long enough, a system will eventually be found in every accessible microstate?

The simple answer is no. To expand on this answer, consider that the most generous projected lifetime of a proton is $10^34$ seconds. There are about $10^44$ plank time units in one second. Even if the system would shuffle every plank time unit for the projected life time of a proton; the number of microstates traversed would still be $10^80$. This number is still \textbf{significantly smaller} than the number of possible microstates.

\section*{HW Problem 3}
A proton has a magnetic moment of $1.396 \frac{e \hbar}{m_p}$. If the magnetic moment is aligned parallel with the z-index (${\mu}_z = \pm \frac{{\mu}_p}{2}$). Treating the proton as a classic paramagnet:

\subsection*{a}
What is the average magnetic energy if $\vec{B} = -.5 \hat{z}$ in units teslas and the temprature $T = 100K$ on average?

As described in equation \textbf{3.31} of the textbook: A magnetic energy for $N$ dipoles is as follows: 
\begin{equation*}
    U = - N \mu B \textrm{tanh} \left( \frac{\mu B}{k_b T} \right)
\end{equation*}

Now, b calculating that $\mu B ~ 1.4 \multiply 10^{-26} J$ and $k_b T = 1.381 \multiply 10^{21} J$ it is found that the value inside of the hyperbolic function is about $1 \multiply 10^{-5}$. This does not change much after applying the hyperbolic tangent, so the average magnetic energy for protons in the specified magnetic field is about $-1.39 \multiply 10^{-31} \frac{J}{proton}$ 

\subsection*{b}
What is its magnetic susceptibility under this alignment? 

Assuming constant volume during the application of the magnetic field, the average magnetic susceptibility is $\chi = \frac{-.5}{-1.39 \multiply 10^{-31}}$ or $3.523 \multiply 10^{30}$

\section*{HW Problem 4}

\end{document}
