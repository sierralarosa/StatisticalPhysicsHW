\documentclass[a4paper,floatfix,nofootinbib]{article}
\usepackage{amsmath,amssymb}
\usepackage{graphicx}
\usepackage{listings}
\graphicspath{ { . } }

\title{Homework 2}
\author{Sierra LaRosa}

\begin{document}

\maketitle

\section*{Problem 2.24}
For a \textit{large} two-state paramagnet the multiplicity function is very sharply peaked about $N_{\uparrow} = N/2$.

\subsection*{a}
Use stirling's approximation to estimate the height of the peak in the multiplicity function.

First, define the multiplicity function for a two-state paramagnet:
\begin{equation*}
    \Omega(N_{\uparrow}) = \binom{N}{N_{\uparrow}}
\end{equation*}

Which as defined by the textbook in equation (2.7), the following can be concluded:
\begin{equation*}
    \Omega_{max} = \frac{N!}{N_{\uparrow} ! N_{\downarrow} ! } 
\end{equation*}
And from the definition of $N_{\uparrow}$ it is shown that: $N_{\downarrow} = N - N/2 = N/2$. When using Stirling's approximation ($N! = N^N e^{-N} \sqrt{2 \pi N} $) is used for the previous equation, it is obvious to see that the following is obtained:
\begin{align*}
    \Omega_{max} &= \frac{N^N e^{-N} \sqrt{2 \pi N}}{{{N/2}^{N/2} e^{-{N/2}} \sqrt{2 \pi {N/2}}}^2} \\
    &= \frac{N^N e^{-N} \sqrt{2 \pi N}}{{N/2}^N e^{-N} \pi N} \\
    &= \frac{N^N e^{-N} \sqrt{2 \pi N}}{\pi N \frac{N^N}{2^N} e^{-N} } \\
    &= 2^N \frac{ \sqrt{2 \pi N}}{\pi N } \\
    \therefore \Omega_{max} &= 2^N \frac{ 2 }{\sqrt{ \pi N }}
\end{align*}

\subsection*{b}
Use the methods of section 2.4 in the textbook to derive a formula for the multiplicity function fo the multiplicity function in the vicinity fo the peak in terms of $x = N_{\uparrow} - \frac{N}{2}$. And check that the formula agrees with \textbf{Problem 2.24a}'s solution. i.e. {x = 0} 

First, it would be beneficial to describe a parameter about the $N_{\uparrow}$ or $\eta = x + \frac{N}{2}$.

Now using the same idea as described in \textbf{Problem 2.24a}'s solution, the following can be shown.
\begin{align*}
   \Omega &= \frac{N!}{\eta ! \left( N - \eta \right) \ ! } \\
   &= \frac{N!}{\left(x + \frac{N}{2} \right) ! \left( N - \left(x + \frac{N}{2} \right) \right) \ ! } \\
    &= \frac{N!}{\left(x + \frac{N}{2} \right) ! \left( \frac{N}{2} - x \right) \ ! }
\end{align*}
And applying Stirling's approximation:
\begin{align*}
    \Omega &= \frac{N^N e^{-N} \sqrt{2 \pi N}}{ \left(x + \frac{N}{2}\right)^\left(x + \frac{N}{2}\right) e^{-\left(x + \frac{N}{2}\right)} \sqrt{2 \pi \left(x + \frac{N}{2}\right)} \times \left( \frac{N}{2} - x \right)^\left( \frac{N}{2} - x \right) e^{-\left( \frac{N}{2} - x \right)} \sqrt{2 \pi \left( \frac{N}{2} - x \right)} } \\
    &= \frac{N^N}{{\left( x + \frac{N}{2} \right)}^{x + \frac{N}{2}} \multiply {\left( \frac{N}{2} - x \right)}^{ \frac{N}{2} - x } } \multiply \sqrt{\frac{N}{2 \pi \left(x + \frac{N}{2} \right) \left( \frac{N}{2} - x \right) }} \\
    &= \frac{N^N}{{\left( x + \frac{N}{2} \right)}^{x + \frac{N}{2}} \multiply {\left( \frac{N}{2} - x \right)}^{ \frac{N}{2} - x } } \multiply \sqrt{\frac{N}{2 \pi \left( \frac{N^2}{4} - x^2 \right) }} \\
    &= \frac{N^N}{{\left( x + \frac{N}{2} \right)}^{\frac{N}{2}} \multiply {\left( x + \frac{N}{2} \right)}^{x} \multiply {\left( \frac{N}{2} - x \right)}^{ \frac{N}{2} } \multiply {\left( \frac{N}{2} - x \right)}^{ -x } } \multiply \sqrt{\frac{N}{2 \pi \left( \frac{N^2}{4} - x^2 \right) }} \\
    &= \frac{N^N}{{\left( x + \frac{N}{2} \right)}^{\frac{N}{2}} \multiply {\left( x + \frac{N}{2} \right)}^{x} \multiply {\left( \frac{N}{2} - x \right)}^{ \frac{N}{2} } \multiply {\left( \frac{N}{2} - x \right)}^{ -x } } \multiply \sqrt{\frac{N}{2 \pi \left( \frac{N^2}{4} - x^2 \right) }} \\
    \ln{\Omega} &= N\ln(N) - \frac{N}{2} \ln \left[ \frac{N^2}{4} - x^2 \right] - x \ln \left[ x + \frac{N}{2} \right] + x \ln \left[ \frac{N}{2} - x \right] + \ln \sqrt{\frac{N}{2 \pi}} - \frac{1}{2} \ln \left(\frac{N^2}{4} - x^2 \right)
\end{align*}
Now it would be beneficial to the solution that the subsititutions:
\begin{align*}
    ln \left[ {\frac{N}{2}}^2 - x^2 \right] &= \ln \left( { \frac{N}{2} }^2 \right) + \ln  \left[ 1 - {\left( \frac{2x}{N} \right)}^2 \right]
    &= 2 \ln \left( \frac{N}{2} \right) + {\left( \frac{2x}{N} \right)}^2
\end{align*}
, and
\begin{align*} 
    ln \left[ \frac{N}{2} \pm \right] &= \ln \left( \frac{N}{2} \right) + \ln \left( 1 \pm \frac{2x}{N} \right)
    &= \ln \left( \frac{N}{2} \right) \pm \frac{2x}{N}
\end{align*}
(The precident for these jumps is dictated in the work done in the set of equations 2.19 in the textbook).
Now pluggin in the previous subsititutions in to the previous work:
\begin{align*}
    \ln \Omega &= N \ln N \ - \ N \ln \frac{N}{2} \ + \ \frac{2 x^2}{N} - x \ln \frac{N}{2} - \frac{2 x^2}{N} + x \ln \frac{N}{2} - \frac{2 x^2}{N} + \ln \sqrt{\frac{N}{2 \pi}} - \ln \frac{N}{2} + \frac{2 x^2}{N^2}
    &= N \ln 2 \ + \ \frac{2 x^2}{N} + \ln \sqrt{\frac{N}{2 \pi}} + \frac{2 x^2}{N^2}
    \therefore \Omega &= 2^N \sqrt{\frac{2}{\pi N}} e^{2x^2 / N}
\end{align*}
This results in a Gaussian function as described in the textbook. And as $x \rightarrow 0$ the closer the answer to \textbf{Problem 2.24a} matches with this result!

\subsection*{c}
How wide is the multiplicity function?

Simple: any Gaussian function falls off of its peak at ${\Omega}_{\pi} = \frac{{\Omega}_max}{e}$. Here that value would be defined as the $\frac{2x_{\pi}^2}{N} = 1$ which becomes $x_{\pi} = \sqrt{\frac{N}{2}}$ or the full width of the peak is $\sqrt{2N}$

\end{document}
